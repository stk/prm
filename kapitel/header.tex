%
% Headerdatei der Diplomarbeit
%
%

\documentclass[
        %pdftex,
		a4paper,
		12pt,
		%oneside,
		%openright,
        %openany, % gegenteil von openright, seite beliebig beginnen, kein effekt bei 'oneside'
		%halfparskip,
%		draft,
%		chapterprefix,%       Kapitel anschreiben als Kapitel
]{scrartcl}

\usepackage{moreverb}

%
% Package f�r Farben im PDF
%
\usepackage{color}

%Deutsche Trennungen, Anf�hrungsstriche und mehr:
\usepackage{german, ngerman}
\usepackage[german]{babel}
%\usepackage[babel,german=quotes]{csquotes}
%Eingabe von �,�,�,� erlauben
\usepackage[latin1]{inputenc}

%Zum Einbinden von Grafiken
\usepackage{graphicx}

%zum benutzen von ps in pdf
\usepackage{ps4pdf}

%Zum malen von bildern
\PSforPDF{
\usepackage{texdraw}
}

%Ein Paket, das die Darstellung von "Text, wie er eingegeben wird"
%erlaubt: Also
%\begin{verbatim} \end{document}\end{verbatim} erzeugt die Ausgabe von
%\end{document} im Typewrites-Style und beendet nicht das Dokument.
\usepackage{verbatim}
%\usepackage{moreverb}

%Source-Code printer for LaTeX
\usepackage{listings}

%Darstellung des Glossars einstellen
%\usepackage[style=super, header=none, border=none, number=none, cols=2,
%						toc=true]{glossary}

%\makeglossary

%
% Paket f�r Links innerhalb des PDF Dokuments
%
\definecolor{LinkColor}{rgb}{0,0,0}
\usepackage[
	pdftitle={RFID Projekt},
	pdfauthor={Sebastian Wiegand, Erik Soehnel, Steffen Kersten},
	pdfcreator={Latex},
	pdfsubject={Zugang per RFID zur FHDW Bibliothek},
	pdfkeywords={RFID Projekt, }]{hyperref}
\hypersetup{colorlinks=true,
	linkcolor=LinkColor,
	citecolor=LinkColor,
	filecolor=LinkColor,
	menucolor=LinkColor,
	pagecolor=LinkColor,
	urlcolor=LinkColor}

